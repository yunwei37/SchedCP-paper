\documentclass[preprint]{article}

%% Define the \sys command for the system name
\newcommand{\sys}{SchedCP\xspace}
%% Define the \agent command for the sched-agent name
\newcommand{\agent}{sched-agent\xspace}


\usepackage{neurips_2025}
% \usepackage[final]{neurips_2025}


\usepackage{listings}     % For ASCII-art / code blocks
\usepackage{booktabs}     % Nicer tables
\usepackage{array}        % Column types
\usepackage{tabularx}     % Automatic column width
\usepackage{enumitem}     % Compact lists



\usepackage{comment}

\usepackage[utf8]{inputenc}
\usepackage[T1]{fontenc}
\usepackage{textcomp}
\usepackage[english]{babel} 
\usepackage{url}
\usepackage{graphicx}


\usepackage{hyperref}       % hyperlinks
\usepackage{amsfonts}       % blackboard math symbols
\usepackage{nicefrac}       % compact symbols for 1/2, etc.
\usepackage{microtype}      % microtypography
\usepackage{xcolor}         % colors




\title{Towards Agentic OS: An LLM Agent Framework for Linux Schedulers}



\author{%
  Yusheng Zheng$^{1}$ \quad
  Yanpeng Hu$^{2}$ \quad
  Andi Quinn$^{1}$ \\
  $^{1}$UC Santa Cruz, CA, USA \quad
  $^{2}$ShanghaiTech University, Shanghai, China \\
  \texttt{\{yzhen165, aquinn1\}@ucsc.edu, huyp@shanghaitech.edu.cn}
}
\sloppy
\begin{document}


\maketitle


\begin{abstract}
Operating system schedulers suffer from a fundamental semantic gap, where kernel policies fail to understand application-specific needs, leading to suboptimal performance. We introduce \sys, the first framework that enables fully autonomous Large Language Model (LLM) agents to safely and efficiently optimize Linux schedulers without human involvement. Our core insight is that the challenge is not merely to \emph{apply} a better LLM, but to architect a decoupled control plane that separates the AI's role of semantic reasoning ("what to optimize") from the system's role of execution ("how to observe and act"). Implemented as Model Context Protocol(MCP) server, \sys provides a stable interface with three key services: a Workload Analysis Engine, an evolving Scheduler Policy Repository, and an Execution Verifier that validates all AI-generated code and configure before deployment with static and dynamic analysis. 

We demonstrate this architecture's power with \agent, a multi-agent system that autonomously analyzes workloads, synthesizes custom eBPF scheduling policies, and deploys them via the sched\_ext infrastructure. Our evaluation shows that SchedCP achieves up to an 1.79x performance improvement, and a 13x cost reduction compared to naive agentic approaches, all while maintaining high success rate. By bridging the semantic gap, SchedCP democratizes expert-level system optimization and represents a step towards creating truly self-optimizing, application-aware operating systems. The code will be open-sourced.
\end{abstract}



\maketitle
\section{Introduction}

Efficient scheduling is critical for performance in multi-tenant HPC clusters and cloud datacenters, where jobs compete for CPUs, GPUs, memory, and I/O resources. Traditional schedulers (e.g. SLURM, PBS, Torque) assume jobs have fixed resource demands and durations. For example, Slurm's batch scheduler allocates resources based on user-specified estimates, leading to long queues and low utilization when workloads vary unexpectedly~\cite{arxiv2401}. In contrast, modern cloud schedulers (e.g. Kubernetes, Borg) allow elastic scaling, but HPC environments lack such dynamic adaptability, often reporting dozens of minutes of average wait time for jobs and low overall resource utilization~\cite{arxiv2401}. Meanwhile, emerging AI-powered workloads (e.g. large-scale training, inference chains) and heterogeneous architectures demand multi-objective scheduling (latency vs throughput vs energy) and fine-grained resource awareness. Manually designing scheduling heuristics or tuning parameters for each scenario is prohibitively expensive and error-prone.

Recent research has explored machine learning for scheduling. Reinforcement learning (RL) agents have been shown to learn scheduling policies directly from job traces. For instance, \emph{Decima} uses RL on graph neural networks to schedule batch jobs in data clusters, improving throughput and makespan over heuristic baselines. \emph{DRAS} employs deep RL for HPC batch scheduling with job reservation and backfilling, achieving up to 45\% reduction in job completion time versus static policies. Other works combine graph neural nets and RL for DAG and resource co-scheduling, yielding significant utilization gains. However, RL approaches require extensive training for each environment and may struggle with new workload patterns or multiple scheduling objectives.

Meanwhile, LLM-based agents have demonstrated remarkable reasoning abilities in complex tasks. Even under zero-shot prompting, LLMs can produce reasonable initial solutions for scheduling problems. For example, an LLM (e.g. GPT-4) can cluster and assign conference papers to sessions, creating draft schedules that are often only a few adjustments away from human quality. In HPC scheduling, recent work shows that an LLM can reason through multi-objective scheduling scenarios (minimize makespan, wait time, etc.) using a ReAct (reason+act) framework, balancing goals without domain-specific training. These results suggest that LLMs can generalize scheduling knowledge from text and math contexts to new workloads.

Our insight is that an LLM agent can automatically analyze a workload and generate a custom scheduler in a high-level form, which is then enacted or fine-tuned by RL. By treating scheduling as an AI planning problem, \sys{} can leverage chain-of-thought reasoning and vast pretraining to capture domain heuristics, while RL handles environment-specific optimization. Concretely, the \sys{} workflow is: (1) \emph{Workload Analysis}: the LLM reads a description of the tasks (program code snippets, performance traces, system constraints) and identifies key properties (e.g. task DAG structure, compute vs I/O intensity, real-time demands). (2) \emph{Policy Synthesis}: using a scheduling DSL, the agent generates a candidate scheduling policy or parameter configuration. (3) \emph{RL Fine-tuning}: an RL loop simulates the proposed scheduler on the workload; performance feedback is used to refine the policy (e.g. adjusting priorities or parameters). This closed loop continues until convergence. Crucially, the LLM's plan serves as a strong initialization that guides RL, requiring far fewer iterations than blind search.

\sys{}'s generality and automatic nature address several challenges:

\begin{itemize}
\item \emph{Diverse Workloads}: By analyzing workload semantics, the LLM tailors scheduling to each scenario, from CPU-bound compute jobs to I/O-heavy pipelines, without manual reprogramming. We evaluate on a broad suite of benchmarks (HPC linear algebra DAGs, multi-stage dataflows, LLM inference jobs) to test generalization.
\item \emph{Complex Objectives}: The system can incorporate multi-objective goals (throughput, latency, fairness) into the agent prompt, letting the LLM reason about trade-offs.
\item \emph{Explainability}: The LLM can produce human-readable justifications for scheduling decisions (via chain-of-thought), aiding debugging and trust in mission-critical settings.
\item \emph{DSL Integration}: We define a small domain-specific language for scheduling policies (inspired by prior scheduler DSLs). The LLM outputs code or configuration in this DSL, which is then compiled into the system. The DSL abstracts low-level details while remaining expressive enough to capture policies.
\end{itemize}

In summary, we contribute: (1) The design of \emph{\sys{}}, a new framework combining LLM reasoning with reinforcement learning to generate dynamic schedulers. (2) A scheduling DSL and Agent-System interface that enables the LLM to emit executable policies while simplifying RL tuning. (3) A comprehensive evaluation on varied workloads showing \sys{} improves scheduling performance over static heuristics and pure-RL baselines, and generalizes across tasks.
\section{Background}
\label{sec:background}

\subsection{Linux Scheduling and sched\_ext}

Linux's default CFS (Completely Fair Scheduler)~\cite{wong2008cfs} implements a one-size-fits-all policy but is unoptimized for diverse workloads ranging from latency-sensitive web services to throughput-oriented batch jobs. While CFS ensures fairness through virtual runtime, modern systems require adaptive scheduling. sched\_ext~\cite{schedext2024}, introduced in Linux 6.12, enables dynamic loading of custom schedulers as eBPF programs without kernel modifications. Built on eBPF~\cite{mccanne1993bpf,gregg2019bpf}, which evolved from packet filtering into a general-purpose in-kernel VM, sched\_ext provides hooks for task enqueueing, CPU selection, load balancing, and idle management. The eBPF verifier ensures safety through static analysis, preventing crashes, invalid memory access, and infinite loops. Production schedulers like scx\_rusty (work-stealing), scx\_layered (hierarchical), and scx\_central (NUMA-aware) demonstrate eBPF can implement complex algorithms matching traditional schedulers with minimal overhead (<1\%).

\subsection{LLMs and Autonomous Agents}

Large language models (LLMs) have transformed software development. Models such as OpenAI’s Codex~\cite{chen2021codex}, GPT-4~\cite{openai2023gpt4}, and Anthropic’s Claude~\cite{anthropic2024claude} can generate complex code from natural language descriptions, automating tasks from code synthesis to debugging. Empirical studies report that GitHub Copilot  helps developers complete coding tasks over 50\% faster on average~\cite{peng2023impact}. 
LLMs have begun to contribute not to development but to system maintenance, such as troubleshooting distributed systems\cite{de2025llm}.
We also have seen a shift from using LLMs purely as code assistants to deploying autonomous LLM-based agents that carry out end-to-end software engineering workflows in 2024-2025. Emerging agent systems like Anthropic’s Claude Code, GitHub’s Copilot Workspace, and Cognition’s Devin aim to handle the entire development cycle — from interpreting requirements and architecting solutions to implementing multi-file code changes, testing, and debugging — with minimal human intervention~\cite{dohmke2024copilotworkspace,sharma2024devin}. Multi-agent frameworks are also being explored, where multiple specialized LLM agents collaborate on software projects~\cite{qian2024chatdev,hong2023metagpt}. However, even the most advanced AI coding tools today primarily serve as developer aids rather than autonomously optimizing low-level system components. This gap suggests an opportunity to leverage LLMs’ semantic understanding to bridge high-level application needs with underlying system-level optimizations. 
\section{Motivation}
\label{sec:motivation}

We motivate our work by examining this semantic gap problem and the practical safety, performance, and cost issues revealed by our experiments.

\subsection{The Semantic Gap Problem}

\textbf{Domain Knowledge Gap between developer and user}: In cloud environments, DevOps engineers configuring Kubernetes lack insight into workload characteristics (latency-sensitive vs. throughput-oriented), resulting in conservative scheduling. Edge and personal devices face worse challenges. Gamers, creative professionals, and office workers lack kernel expertise for optimization. LLMs can bridge this gap by understanding high-level workload patterns from source code and deployment artifacts, and translating them into concrete scheduling policies.

\textbf{Technical Complexity of Scheduler Development}: Linux scheduler development requires mastering kernel programming with lock-free structures, eBPF verification constraints, and CPU/NUMA architectures. This steep learning curve limits innovation to few kernel experts. LLMs with pre-trained domain knowledge enable rapid and automated scheduler development without human expertise.

\textbf{Dynamic Workload Adaptation}: Modern workloads exhibit complex phase behavior: ML training alternates between compute-intensive forward propagation and communication-heavy gradient synchronization; web traffic varies by orders of magnitude daily; build system parallelism changes with dependencies. Manual reconfiguration and optimization cannot keep pace, while AI agents can adapt policies in real-time, operating 24/7 without breaks, much more availability than human experts.

\subsection{Economic Viability of AI-Driven Scheduler Optimization}

Scheduler optimization becomes economically viable when generation costs are lower than CPU time savings. For example, with \$0.45 AI generation LLM API cost and 20\% performance improvement, a 10-hour workload on a \$2/hour instance already breaks even (\$0.45 cost vs \$4 savings). This fundamentally changes scheduler economics: traditionally, high manual engineering costs meant custom schedulers were only justified for large-scale cloud workloads running on hundreds of machines for months, but AI assistance now makes it economical to optimize even short-lived workloads like CI/CD pipelines, batch jobs, or individual builds running for just hours or days on single machines, democratizing performance optimization for previously uneconomical use cases.

\subsection{Motivation Experiment}

We tested Claude Code\cite{claudecode}, the state of the art LLM agent, with "write a FIFO scheduler in eBPF" from an empty folder, with all permissions and bash access. Of three attempts, only one succeeded. The second attempt produced pseudo-code after 6 minutes trying, and the third generated a scheduler tracer instead after 8 minutes development. The successful generation required 33 minutes, 221 LLM API calls, and 15+ iterations, costing \$6 (vs. 5 minutes typically for an expert developer). The generated code, for some workloads, exhibited poor quality with excessive overhead, performing worse than CFS. The agent required root access, could crash the system during testing, and lacked gradual rollout mechanisms, which also raises safety concerns.

\subsection{Challenges in Applying LLM Agents to Schedulers}

Our experiments reveal critical challenges for AI-driven scheduler optimization, especially when fully automated: \textbf{Performance}: How do we ensure AI-generated or configured schedulers outperform existing ones rather than degrading performance? \textbf{Safety}: How do we prevent kernel crashes, soft-lockups, stalls, or starvation while maintaining stability? How can we ensure only minimal privilege needed when development and deployment? \textbf{Efficiency}: The 33-minute generation time and the \$6 cost must drop for practical deployment.

\input{sections/design\_impl}
\section{Evaluation}

\subsection{Experimental Setup}

\subsubsection{Hardware Configuration}
Our evaluation uses a diverse hardware testbed representing modern server deployments. The primary test system has a 32-core AMD EPYC 7543 processor with 256GB of DDR4-3200 RAM to stress-test scheduler implementations. The EPYC platform's chiplet architecture, with multiple CCX (Core Complex) units connected via Infinity Fabric, presents challenges for NUMA-aware scheduling that our AI agents must handle. Storage uses enterprise-grade NVMe SSDs with sustained read speeds exceeding 5GB/s, eliminating I/O bottlenecks and focusing tests on scheduler behavior. The 10Gbps ethernet connectivity enables distributed workload testing for modern microservice architectures. All systems run Linux 6.12 with sched\_ext enabled, compiled with frame pointers for accurate profiling and standard distribution settings for reproducibility.

\subsubsection{AI Agent Configuration}
We selected multiple AI agents to evaluate our framework's generality and different model capabilities. Claude Code (Opus 4) serves as our primary agent for its code generation abilities and system-level understanding. We compare its performance against GPT-4 and Gemini Pro to understand how model capabilities affect scheduler quality. For cost-sensitive deployments, we evaluate Llama-3 70B running locally on A100 GPUs, measuring trade-offs between API costs and generation quality. Our reinforcement learning framework uses Proximal Policy Optimization (PPO) with custom reward functions balancing performance improvement, stability, and resource efficiency. The RL agent operates with a context window of the last 100 scheduling decisions to capture workload patterns while maintaining computational efficiency.

\subsubsection{Workload Selection}
Our workloads span modern computing tasks, each stressing different scheduler behaviors. The Linux kernel build (make -j32) represents parallel compilation with complex inter-task dependencies and varying task durations. schbench serves as our latency-sensitive benchmark, simulating request-response patterns in web services with configurable think times and message sizes. For data-intensive workloads, we run TPC-H queries on a 100GB dataset, testing the scheduler's ability to balance compute and memory bandwidth. Video processing through FFmpeg 4K transcoding tests sustained CPU utilization with predictable patterns. Git operations on the Linux kernel repository (1M+ commits) stress the scheduler with rapid task creation and destruction. The Chromium browser test suite (100,000+ unit tests) evaluates scheduling fairness and efficiency under extreme task counts. PyTorch distributed training jobs test the scheduler's coordination of tightly coupled parallel tasks with synchronization barriers.

\subsection{Research Questions and Results}

\subsubsection{RQ1: Can LLM agents effectively configure existing schedulers?}

Our first research question examines whether LLM agents can understand workload characteristics and map them to appropriate scheduling strategies. This represents our system's simplest use case—using existing, well-tested schedulers with AI-driven configuration rather than generating new code. Correct scheduler selection and configuration demonstrates that LLMs can bridge the gap between high-level workload descriptions and low-level scheduling parameters.

\textbf{Methodology}: We present workload descriptions to the LLM agent in natural language, mimicking system administrator requirements. For each workload, we provide task characteristics, performance goals, and system constraints. The agent selects a scheduler from our library and configures its parameters. We measure performance improvement against the baseline Linux CFS scheduler. Each experiment runs five times for statistical significance, reporting median values to account for system variance.

\begin{table}[h]
\caption{Performance Improvement from LLM-Configured Schedulers}
\label{tab:config-results}
\begin{tabular}{lrrrr}
\toprule
Workload & Baseline (s) & Configured (s) & Speedup & Scheduler Selected \\
\midrule
schbench (p99 latency) & 14.2ms & 7.1ms & 2.0x & scx\_layered \\
Linux kernel build & 312s & 173s & 1.8x & scx\_rusty \\
TPC-H Q1 & 45.3s & 31.2s & 1.45x & scx\_central \\
Video transcode & 521s & 412s & 1.26x & scx\_rusty \\
\bottomrule
\end{tabular}
\end{table}

\textbf{Key Findings}: The LLM showed sophisticated workload understanding and scheduler selection. The agent distinguished between latency-sensitive workloads like schbench and throughput-oriented tasks like kernel compilation, selecting appropriate schedulers. For schbench, it chose scx\_layered with minimal time slices and aggressive priority boosting for waking tasks, achieving 2x reduction in p99 latency. For kernel builds, it selected scx\_rusty with work-stealing enabled and tuned the steal threshold based on core count. The agent recognized that TPC-H queries benefit from NUMA-aware scheduling and configured scx\_central accordingly. Performance improvements ranged from 26\% to 100\% across workloads. When asked to explain its choices, the agent's reasoning matched expert analysis in 85\% of cases, showing true understanding of scheduling principles beyond pattern matching.

\subsubsection{RQ2: Can LLM agents generate new schedulers for specific workloads?}

The true test of our system is generating new schedulers tailored to specific workload requirements. This moves beyond configuration to code synthesis, requiring the LLM to understand scheduling theory and implementation details. We focus on workloads where existing schedulers are insufficient, forcing the agent to innovate rather than adapt existing solutions.

\textbf{Batch Processing Optimization Experiment}: We selected batch processing workloads because they present challenges poorly addressed by general-purpose schedulers. These workloads, common in build systems and data analytics pipelines, have tasks with variable execution times—some complete in milliseconds while others run for minutes. Traditional fair-share schedulers like CFS waste time on suboptimal task ordering. We challenged the LLM agent to analyze these patterns and generate specialized schedulers that minimize completion time or average wait time based on the optimization goal.

\begin{table}[h]
\caption{AI-Generated Scheduler Performance on Batch Workloads}
\label{tab:batch-results}
\begin{tabular}{lrrr}
\toprule
Workload & Default CFS & AI-Generated & Strategy Used \\
\midrule
Compilation (makespan) & 100\% & 68\% (-32\%) & Longest Job First \\
Unit tests (avg wait) & 100\% & 55\% (-45\%) & Shortest Job First \\
Data analytics & 100\% & 71\% (-29\%) & Hybrid approach \\
Git gc (large repo) & 100\% & 64\% (-36\%) & Dependency-aware \\
\bottomrule
\end{tabular}
\end{table}

\textbf{Insights from Generated Code}: The AI-generated schedulers show sophisticated understanding of scheduling theory in practice. For unit tests minimizing average wait time, Claude Opus correctly implemented Shortest Job First (SJF) scheduling, a theoretically optimal strategy. The implementation maintained a min-heap of tasks sorted by expected duration and used historical data to predict task lengths. For compilation workloads where total makespan matters, the agent implemented Longest Job First (LJF) scheduling, maintaining parallelism by avoiding situations where only small tasks remain. For complex build systems, the agent generated a dependency-aware scheduler that analyzes the build graph and prioritizes critical path tasks. The code included graph analysis to identify bottlenecks and dynamic reprioritization as tasks complete. Performance improvements of 29-45\% across batch workloads show these theoretical insights translate to practical benefits.

\subsubsection{RQ3: What is the cost and efficiency of AI-driven scheduler generation?}

Cost efficiency is critical for practical adoption of AI-driven optimization. Our motivation experiments showed prohibitive costs for naive approaches, making it essential to demonstrate economically viable operation. We analyze resource requirements across time, API calls, and monetary costs, comparing our optimized approach against naive baselines and traditional human development.

\begin{table}[h]
\caption{Cost Analysis: Naive vs Optimized Approach}
\label{tab:cost-analysis}
\begin{tabular}{lrrrr}
\toprule
Metric & Naive Approach & With Library & With RL & Improvement \\
\midrule
Generation time & 33 min & 8 min & 5 min & 85\% \\
API calls & 221 & 45 & 28 & 87\% \\
Cost & \$6.00 & \$1.20 & \$0.75 & 88\% \\
Success rate & 65\% & 92\% & 95\% & +30pp \\
\bottomrule
\end{tabular}
\end{table}

\textbf{Cost Reduction Strategies}: Our framework achieves cost reductions through multiple optimizations. The scheduler library serves as institutional memory, eliminating redundant generation—60\% of requests are satisfied through library adaptation rather than full generation. Semantic search through our pattern database provides relevant examples and proven patterns, cutting average iteration count from 15+ to under 3. The reinforcement learning component improves first-attempt success rates by learning from successes and failures, reaching 95\% success rate compared to 65\% for naive approaches. Caching API responses for documentation queries and code analysis reduces API calls by 40\%. Template reuse for common scheduler patterns lets the AI focus on customization rather than boilerplate generation. These optimizations achieve 85-88\% cost reduction while improving output quality, making AI-driven scheduler optimization economically competitive with human development.

\subsubsection{RQ4: How much can RL improve performance after initial generation?}

\begin{figure}[h]
\centering
\fbox{\parbox{0.9\columnwidth}{\centering
\vspace{1.5cm}
RL Performance Improvement Over Time\\
(Showing iterative improvement through feedback loop)\\
\vspace{1.5cm}
}}
\caption{RL Performance Improvement Over Time}
\label{fig:rl-improvement}
\end{figure}

LLMs generate reasonable initial implementations but cannot observe runtime behavior or learn from deployment experience. Our reinforcement learning component addresses this by continuously optimizing schedulers based on performance feedback. This research question quantifies the benefits of RL-based refinement beyond initial LLM generation.

\textbf{Experimental Protocol}: We isolate reinforcement learning impact through controlled experiments. First, the LLM generates an initial scheduler for each workload, establishing baseline performance. Then, the RL component runs 100 optimization episodes. Each episode: (1) adjusts scheduler parameters or code based on current policy, (2) runs the modified scheduler on the target workload, (3) computes rewards based on performance metrics, and (4) updates the policy using PPO. We measure performance improvements at 10-episode intervals to understand convergence. The reward function balances primary performance metrics (latency or throughput), stability (variance), and resource efficiency (CPU overhead).


\textbf{Results}: Reinforcement learning substantially refines LLM-generated schedulers. Starting from an improved baseline (LLM-generated scheduler showing 15\% improvement over CFS), RL optimization achieves additional 10-12\% performance gain, bringing total improvement to 25-27\%. Learning curves show rapid improvements in the first 20 episodes as RL discovers obvious parameter optimizations, followed by gradual refinement exploring subtle trade-offs. Convergence occurs within 50-60 episodes, after which training yields diminishing returns. RL improvements come from: fine-tuning parameters like time slice lengths and migration thresholds to match workload behavior, discovering workload-specific patterns not in initial generation, and optimizing edge cases absent from LLM training data. RL maintains stability—no cases degraded performance below the LLM baseline due to conservative policy updates and safety constraints.

\subsubsection{RQ5: How effectively can LLMs understand workloads?}

AI-driven scheduler optimization depends on LLMs understanding workload characteristics from high-level descriptions and system observations. This research question evaluates whether LLMs can accurately categorize workloads and extract relevant features for scheduling decisions.

\begin{table}[h]
\caption{Workload Classification Accuracy}
\label{tab:workload-understanding}
\begin{tabular}{lrrr}
\toprule
Workload Category & Correct Classifications & Total & Accuracy \\
\midrule
CPU-intensive & 47 & 50 & 94\% \\
I/O-bound & 43 & 50 & 86\% \\
Memory-intensive & 41 & 50 & 82\% \\
Latency-critical & 48 & 50 & 96\% \\
Batch processing & 45 & 50 & 90\% \\
\midrule
Overall & 224 & 250 & 89.6\% \\
\bottomrule
\end{tabular}
\end{table}

\textbf{Classification Features Used by Agent}: The LLM's decision-making reveals sophisticated feature extraction matching expert analysis. The agent examines system call patterns, inferring that high read/write ratios indicate I/O-bound workloads while syscall-light applications are CPU-intensive. It analyzes CPU utilization, distinguishing sustained high utilization (compute-bound) from bursty patterns (interactive). Memory access patterns identify cache-sensitive workloads benefiting from NUMA-aware scheduling. The agent analyzes task creation and lifetime distributions—short-lived tasks with high creation rates suggest compilation or test workloads, while long-lived tasks indicate services or batch processing. I/O wait time percentages signal bottlenecks for adjusting scheduling priorities. The 89.6\% overall classification accuracy shows LLMs can bridge the gap between high-level workload descriptions and low-level system behaviors, validating natural language as the interface for scheduler optimization.

\subsection{Case Studies}

\subsubsection{Case Study 1: Linux Kernel Compilation}

Linux kernel compilation stresses multiple aspects of scheduler design. The build process involves thousands of compilation units with complex interdependencies, creating a challenging scheduling problem. This case study demonstrates how our AI-driven approach discovers and implements optimizations difficult for human administrators to achieve manually.

The agent's analysis showed understanding of the compilation process. It identified kernel builds have heavy parallelism with tasks ranging from sub-second header processing to multi-second compilation of large source files. The agent recognized dependencies—linking operations wait for object files, creating synchronization points that become bottlenecks. It noted the mix of CPU-intensive compilation and I/O operations for reading source files and writing objects, requiring balanced scheduling.

The generated scheduler implemented three innovations. First, dependency-aware prioritization analyzes the makefile graph to identify critical path tasks, ensuring bottleneck operations receive CPU priority. Second, dynamic work stealing addresses load imbalance in compilation workloads, with idle CPUs pulling tasks from overloaded cores' queues. Third, separate queues for linking versus compilation tasks prevent linking operations from being starved by numerous small compilation tasks.

Results: 80\% speedup reducing build time from 312 seconds to 173 seconds. This improvement comes from better CPU utilization (average 95\% vs 72\% for CFS), reduced waiting for critical path tasks, and intelligent task ordering maintaining parallelism throughout the build.

\subsubsection{Case Study 2: Interactive Latency Optimization}

Interactive latency optimization differs from throughput-oriented workloads. The schbench benchmark simulates request-response patterns of web services and databases, where tail latency impacts user experience. This case study shows how our AI agent identifies and optimizes latency-sensitive characteristics without explicit programming.

The agent's analysis revealed patterns informing its optimization strategy. It identified periodic wake-up patterns of request processing, with threads sleeping between requests and requiring rapid response for new work. The agent recognized latency-sensitivity through metrics showing p99 latency matters more than average throughput. It observed traditional preemption-based fair scheduling introduces unnecessary latency for waking tasks waiting in runqueues.

The generated scheduler implemented targeted optimizations for latency reduction. First, it reduced scheduler tick frequency from 250Hz to 100Hz, minimizing interruption of running tasks while maintaining responsiveness. Second, it implemented strong CPU affinity for hot threads, ensuring tasks wake on CPUs where data remains cache-warm. Third, it added aggressive priority boosting for waking tasks, allowing immediate preemption rather than waiting for the next scheduling quantum.

These optimizations achieved 50\% reduction in p99 latency, from 14.2ms to 7.1ms. Improvement comes from reduced wake-to-run latency (average 45$\mu$s vs 180$\mu$s for CFS) and better cache utilization through affinity. The agent maintained fairness for long-running background tasks through a decay mechanism gradually reducing priority boost over time, showing understanding of latency optimization trade-offs.


\section{Related Work}

Our work stands at the intersection of multiple rapidly evolving research areas: reinforcement learning for system optimization, AI applications in operating systems, the emerging capabilities of large language models for code generation, and the revolutionary potential of autonomous AI agents. We position our contributions relative to these fields, highlighting both how we build upon existing work and where we make fundamental advances.

\subsection{RL-based Scheduler Optimization}

The application of reinforcement learning to scheduling problems has emerged as one of the most promising directions in systems research, with several landmark papers demonstrating the potential for learned policies to outperform traditional heuristics. However, these approaches face fundamental limitations that motivate our LLM-based approach.

\textbf{Decima}~\cite{mao2019decima}, presented at SIGCOMM 2019, represents a watershed moment in applying deep learning to systems problems. The system uses graph neural networks to learn job scheduling policies for datacenter environments, reasoning about complex dependencies between tasks in distributed computing jobs. By representing jobs as directed acyclic graphs and using a novel neural architecture that can process variable-sized inputs, Decima achieves remarkable results—reducing average job completion time by up to 40\% compared to traditional heuristics. However, Decima's approach requires extensive training for each specific workload type, often needing millions of scheduling decisions before converging to good policies. More critically, it operates purely on statistical patterns in job structures without understanding what the jobs actually do or why certain scheduling decisions might be beneficial. This limitation becomes apparent when workloads exhibit rare but important patterns that weren't well-represented in training data.

\textbf{Firm}~\cite{qiu2020firm}, published at OSDI 2020, extends the RL paradigm to address multi-resource cluster scheduling while maintaining fairness constraints—a notoriously difficult problem in distributed systems. The system models resource allocation as a multi-agent reinforcement learning problem, where each job acts as an agent competing for cluster resources. Firm's innovation lies in its ability to learn policies that balance efficiency with fairness, preventing scenarios where some jobs are starved while others monopolize resources. The system achieves impressive results in production deployments, improving cluster utilization by 25-30\% while maintaining strict fairness guarantees. However, like Decima, Firm cannot reason about application semantics or understand why certain workloads might benefit from specific scheduling strategies. The learned policies remain black boxes that provide no explanation for their decisions, making debugging and trust difficult in production environments.

\textbf{Recent advances} in RL-based scheduling have pushed the boundaries further. Multi-objective RL schedulers like MrSch~\cite{zhang2024mrsch} simultaneously optimize for multiple resources including CPU, memory, and network bandwidth, achieving up to 48\% performance improvements in heterogeneous clusters. These systems use sophisticated neural architectures including attention mechanisms and transformer models to capture complex interactions between different resource types. Other work has explored hierarchical RL for multi-level scheduling decisions and meta-learning approaches that can quickly adapt to new workload types. While these advances demonstrate the continued potential of RL for scheduling, they also highlight a fundamental limitation: the inability to leverage domain knowledge or understand application intent.

\textbf{Key distinction}: The fundamental difference between our approach and pure RL methods lies in semantic understanding. While RL-based schedulers must learn everything from scratch through trial and error, our LLM-based system can immediately apply knowledge from its training on vast amounts of code and documentation. When faced with a compilation workload, for instance, an RL system sees only task durations and dependencies, while our system understands that it's looking at a build process and can apply known optimizations like prioritizing tasks on the critical path. We combine the best of both worlds—using LLMs for initial understanding and generation, then applying RL for workload-specific optimization.

\subsection{AI for Systems}

The broader application of artificial intelligence to systems problems has evolved from simple parameter tuning to fundamental reimagining of core system components. This progression provides important context for understanding how our work represents the next logical step in this evolution.

\textbf{Learned Indexes} represent one of the most radical applications of machine learning to systems, as demonstrated by Kraska et al.~\cite{kraska2018learned} in their influential SIGMOD 2018 paper. The key insight is that database indexes are fundamentally models that map keys to positions, and neural networks can learn these mappings more efficiently than traditional B-trees for certain data distributions. By replacing B-trees with small neural networks, learned indexes achieve up to 70\% memory savings and 3x lookup performance improvements. This work sparked a renaissance in rethinking fundamental system components through the lens of machine learning. Subsequent research has extended learned indexes to handle updates, developed hybrid approaches that combine traditional and learned structures, and applied similar ideas to other data structures like bloom filters and hash tables. The success of learned indexes demonstrates that AI can improve even the most fundamental and well-studied system components.

\textbf{Query Optimization} represents another successful application domain where AI techniques have shown significant promise. Neo~\cite{marcus2019neo}, presented at VLDB 2019, uses deep reinforcement learning to optimize database query execution plans. Traditional query optimizers rely on cost models that are notoriously inaccurate, leading to suboptimal plan choices. Neo learns from actual query execution times to build more accurate cost models and make better optimization decisions. The system achieves up to 2x performance improvements on complex analytical queries. More recent work has explored using transformer models to understand query structure and predict optimal join orders, and applying multi-armed bandits to adaptively choose between different query execution strategies. These successes in query optimization demonstrate AI's ability to improve complex decision-making in systems where traditional heuristics fall short.

\textbf{Configuration Tuning} has emerged as a practical application area where AI provides immediate value. OtterTune~\cite{vanaken2017ottertune}, developed at CMU, uses Gaussian Process regression and collaborative filtering to automatically tune database configuration parameters. The system addresses a critical pain point—modern databases have hundreds of configuration knobs, and even expert DBAs struggle to find optimal settings. OtterTune learns from previous tuning sessions across different workloads to recommend configurations for new deployments, achieving performance within 94\% of expert DBAs with minimal training data. Similar approaches have been applied to web server configuration, container resource allocation, and distributed system parameters. While these tools demonstrate AI's practical value, they remain limited to optimizing existing parameters rather than creating new algorithms or implementations.

\textbf{Our contribution} represents a fundamental leap beyond these existing applications. While learned indexes replace specific data structures and configuration tuners optimize parameters, we use AI to generate entire system components—complete schedulers with new algorithms tailored to specific workloads. This moves from AI as an optimization tool to AI as a system designer. Our approach also differs in leveraging large language models' semantic understanding rather than purely statistical learning. When OtterTune sees a configuration parameter, it treats it as an opaque number to optimize. When our system sees a scheduler parameter, it understands its semantic meaning and how it affects system behavior. This semantic understanding enables more intelligent and explainable optimizations.

\subsection{LLMs in Systems and Code Generation}

The emergence of large language models with sophisticated code understanding and generation capabilities has opened unprecedented opportunities for automated system optimization. These models, trained on vast corpora of code and documentation, possess knowledge that spans from high-level algorithms to low-level implementation details, making them uniquely suited for systems tasks that traditionally required years of human expertise.

\textbf{Code Generation} capabilities have advanced dramatically with systems like Codex~\cite{chen2021codex} (powering GitHub Copilot), GPT-4, and Claude demonstrating remarkable ability to generate complex systems code. GitHub Copilot has transformed software development by providing context-aware code suggestions that often match or exceed human-written code quality. Studies show that developers using Copilot complete tasks 55\% faster on average, with the tool being particularly effective for boilerplate code and common patterns. However, these tools are designed as developer assistants rather than autonomous agents. They excel at completing code snippets and implementing well-defined functions but cannot independently design and implement entire system components. More importantly, they lack the feedback loop necessary for optimization—they generate code based on patterns in training data but cannot observe runtime behavior and iterate accordingly. Recent advances have pushed capabilities further, with models generating entire applications from specifications, but the focus remains on functional correctness rather than performance optimization.

\textbf{Systems Understanding} represents another crucial capability where LLMs have shown surprising competence. Recent work by Wang et al.~\cite{wang2024llmsys} demonstrates that LLMs can analyze and explain complex system behaviors, debugging performance issues and suggesting optimizations. Their evaluation shows that GPT-4 can correctly diagnose 78\% of performance problems in distributed systems when provided with logs and metrics. LLMs have also proven capable of understanding kernel code, with studies showing they can explain Linux kernel functions with accuracy comparable to experienced developers. This understanding extends to performance implications—models can predict which code changes are likely to improve or degrade performance based on patterns learned from millions of code examples. However, existing applications have focused on analysis and explanation rather than synthesis. While an LLM can explain why a scheduler might be inefficient, prior to our work there were no systems that could automatically generate improved implementations.

\textbf{Mapper Generation} by Wei et al.~\cite{wei2024mapper} represents the closest prior work to our approach. Their system uses LLMs to generate mappers that optimize parallel program execution on heterogeneous hardware. By representing the mapping problem as a domain-specific language (DSL) generation task, they enable LLMs to produce code that assigns computation to different processing units (CPUs, GPUs, accelerators) based on workload characteristics. The system achieves impressive 3.8× speedups on average across various parallel workloads. The key insight is that LLMs can understand both the high-level structure of parallel algorithms and the low-level details of hardware capabilities, bridging the semantic gap that makes manual optimization difficult. However, mapper generation operates in a more constrained domain than kernel scheduler generation. Mappers work with well-defined parallel patterns and clear performance models, while schedulers must handle arbitrary workloads with complex, emergent behaviors. Our work extends this approach to the significantly more challenging domain of kernel schedulers, where safety requirements are stricter, performance implications more subtle, and the interaction with system state more complex.

\subsection{sched\_ext and eBPF Schedulers}

The introduction of sched\_ext represents a paradigm shift in Linux kernel development, enabling safe, dynamic scheduler implementation for the first time in the kernel's history. This framework provides the essential foundation that makes our AI-driven approach possible, offering both the flexibility needed for experimentation and the safety guarantees required for production deployment.

\textbf{The sched\_ext Framework} fundamentally changes how schedulers can be developed and deployed. Traditional Linux scheduler development required modifying kernel source code, recompiling the entire kernel, and rebooting systems—a process that could take hours and risk system stability. With sched\_ext, schedulers are implemented as BPF programs that can be loaded and unloaded dynamically without system restart. The framework provides comprehensive hooks into the scheduling subsystem, allowing BPF programs to control task enqueueing, CPU selection, load balancing, and idle CPU management. This flexibility enables implementation of novel scheduling algorithms that would be impractical with traditional kernel development. Performance overhead is minimal—typically less than 1\% for production workloads—making sched\_ext suitable for performance-critical environments. The framework has already gained significant adoption, with major technology companies deploying custom schedulers in production to optimize for their specific workloads.

\textbf{Production Schedulers} developed for sched\_ext demonstrate the framework's capability to support sophisticated scheduling policies. scx\_rusty implements a work-stealing scheduler with advanced load balancing, achieving better performance than CFS for many parallel workloads. The scheduler uses per-CPU queues with periodic work stealing to balance load, implementing NUMA-aware stealing policies that minimize cross-socket traffic. scx\_layered provides hierarchical scheduling with cgroup awareness, allowing different scheduling policies for different application classes. This scheduler has proven particularly valuable in cloud environments where multiple tenants require isolation and different performance guarantees. scx\_central implements a centralized scheduling approach optimized for NUMA systems, where a single scheduling queue can actually improve performance by making better global decisions. These production schedulers validate that eBPF is capable of implementing complex scheduling algorithms with performance matching or exceeding traditional kernel schedulers.

\textbf{Safety Guarantees} provided by the eBPF verifier address one of the most critical challenges in dynamic scheduler generation. The verifier performs exhaustive static analysis of BPF programs before loading, ensuring they cannot crash the kernel, access invalid memory, or enter infinite loops. This verification includes checking that all memory accesses are bounds-checked, all loops have provable termination conditions, and all kernel helper functions are called with valid arguments. The verifier's conservative approach means that some correct programs may be rejected, but it guarantees that accepted programs are safe. This safety mechanism is crucial for our AI-driven approach—without it, automatically generated scheduler code could pose unacceptable risks to system stability. The verifier acts as a safety net that allows us to experiment with AI-generated code while maintaining production-level reliability.

\textbf{Our innovation} builds upon sched\_ext's foundation to add intelligence to the scheduling layer. While sched\_ext provides the mechanism for safe, dynamic scheduler implementation, it still requires human experts to design and implement scheduling algorithms. Our contribution is automating this process—using AI to understand workload requirements, generate appropriate scheduling algorithms, and implement them as eBPF programs. This combination of sched\_ext's flexibility and safety with AI's pattern recognition and code generation capabilities enables a new paradigm where schedulers can be automatically optimized for specific workloads. We are the first to demonstrate that AI agents can successfully generate production-quality kernel schedulers, opening new possibilities for adaptive system optimization.

\subsection{AI Agents and Autonomous Systems}

The years 2024-2025 mark an inflection point in AI capabilities, with the emergence of truly autonomous agents capable of complex software engineering tasks. These agents represent a qualitative leap from previous AI systems, combining multiple capabilities into cohesive systems that can work independently on challenging technical problems.

\textbf{Claude Code}, along with similar autonomous agents like GitHub Copilot Workspace and Devin, demonstrates unprecedented software development capabilities. These agents go far beyond simple code completion to engage in complete software engineering workflows. Claude Code can understand complex requirements expressed in natural language, architect solutions, implement code across multiple files, write tests, debug failures, and iterate based on results—all without human intervention. In benchmarks, these agents successfully complete real-world programming tasks that previously required hours of human developer time. The key innovation is the integration of multiple AI capabilities: natural language understanding for requirements, code generation for implementation, program analysis for debugging, and planning algorithms for managing complex multi-step tasks. Our motivation experiments with Claude Code revealed both its potential and current limitations for systems programming, informing our framework design to maximize strengths while mitigating weaknesses.

\textbf{AIOS}~\cite{mei2024aios} represents complementary research that proposes fundamental OS changes to support AI agents as first-class citizens. Their vision includes LLM-aware process scheduling that understands the unique requirements of AI workloads, such as large memory footprints and bursty computation patterns. The system provides new abstractions for AI agents including semantic memory spaces, natural language system calls, and AI-specific resource management. While our work focuses on using AI agents to optimize traditional OS components, AIOS explores how operating systems themselves must evolve to support AI workloads. The two approaches are synergistic—our AI-optimized schedulers could run within an AIOS environment, while AIOS could use our techniques to automatically optimize its own scheduling policies. This convergence points toward a future where AI and operating systems are deeply integrated at multiple levels.

\textbf{Agent Capabilities} have expanded dramatically to encompass the full software development lifecycle. Modern agents combine multiple specialized models and tools: code generation models trained on vast repositories, testing frameworks that automatically generate test cases, debugging tools that can trace execution and identify bugs, and planning systems that break complex tasks into manageable steps. These agents can now handle tasks requiring deep understanding of system constraints, performance implications, and correctness requirements. They can read and understand existing codebases, identify optimization opportunities, and implement improvements while maintaining compatibility. The agents' ability to iterate based on feedback is particularly crucial—they can run tests, observe failures, and modify their approach accordingly. This closed-loop capability is essential for our scheduler optimization use case, where performance can only be validated through actual execution. By leveraging these comprehensive agent capabilities, we can automate scheduler development tasks that previously required expert kernel developers.

\subsection{Positioning Our Work}

Our work is unique in several ways:

\begin{enumerate}
\item \textbf{First to use LLM agents for OS scheduler generation}: While others use ML for scheduling decisions, we generate entire schedulers.

\item \textbf{Semantic Understanding}: Unlike pure RL approaches, our system understands workload intent and requirements.

\item \textbf{Production-Ready}: Built on sched\_ext, our generated schedulers can run in production Linux systems.

\item \textbf{Self-Evolving}: Our scheduler library grows through experience, unlike static ML models.

\item \textbf{Safety-First Design}: We address the unique challenges of AI-generated kernel code through comprehensive safety mechanisms.
\end{enumerate}

\subsection{Limitations of Existing Approaches}

A critical analysis of existing approaches reveals fundamental limitations that motivate our research direction. Each category of current solutions faces inherent constraints that prevent them from achieving the vision of truly adaptive, intelligent system optimization.

\textbf{RL-based methods}, despite their successes, face several fundamental limitations that become apparent in production deployments. The most significant is the extensive training requirement—Decima requires millions of scheduling decisions before converging to good policies, making it impractical for workloads that change frequently or new deployment scenarios. This training is not transferable; a policy learned for one workload type provides little benefit for another, necessitating retraining from scratch. The inability to leverage domain knowledge means that RL systems must rediscover well-known scheduling principles through trial and error, wasting computational resources and time. The policies produced are typically neural networks that provide no explanation for their decisions, making debugging nearly impossible when things go wrong. Most critically, RL methods miss semantic optimization opportunities—they cannot understand that a workload represents compilation and apply known optimizations for build systems. They see only statistical patterns in task durations and miss the higher-level structure that could inform better scheduling decisions.

\textbf{Traditional auto-tuning} approaches, while practical and widely deployed, operate within narrow constraints that limit their effectiveness. These systems can only optimize existing parameters rather than create new algorithms or implementation strategies. OtterTune might find the optimal buffer pool size for a database, but it cannot invent a new buffer management algorithm better suited to the workload. The requirement for manual feature engineering means that optimization is limited to dimensions that humans have already identified and exposed as tunable parameters. Many optimization opportunities exist in implementation choices that are not exposed as parameters. Adaptation to new workloads is slow because these systems must observe sufficient samples to build accurate models, and they cannot leverage understanding of workload semantics to accelerate this process. The fundamental limitation is that auto-tuning operates within the design space defined by human developers rather than exploring new possibilities.

\textbf{Static schedulers}, including the default Linux CFS, embody a one-size-fits-all philosophy that becomes increasingly problematic as workload diversity grows. These schedulers cannot adapt to workload-specific requirements without manual intervention, missing significant optimization opportunities. When new workload patterns emerge—such as the rise of microservices or machine learning training—schedulers require manual updates that may take years to develop and deploy. The pace of innovation is fundamentally limited by developer resources; even when optimization opportunities are known, implementing them requires scarce kernel development expertise. Static schedulers must make conservative choices that work reasonably well for all workloads rather than optimizing aggressively for specific patterns. This leads to a lowest-common-denominator approach that leaves significant performance on the table for specialized workloads.

Our LLM agent-based approach addresses these limitations through a fundamentally different architecture. By combining semantic understanding from LLMs, code generation capabilities, and continuous learning through RL, we create a system that can understand workload intent, generate specialized implementations, and improve through experience. This unified framework breaks through the constraints that limit existing approaches, enabling a new generation of adaptive, intelligent system optimization.
\section{Future Work and Impact}

\subsection{Extended Scope}

Our framework extends to memory management with custom page replacement policies, NUMA optimization, I/O scheduling for NVMe SSDs, and power management through intelligent DVFS policies. Cross-component optimization coordinating CPU, memory, I/O, and power decisions holistically presents significant opportunities. Research challenges include formal verification of AI-generated schedulers, understanding performance improvement bounds, and leveraging AI advances like multi-modal understanding and million-token context windows.

\subsection{Broader Impact}

This work democratizes OS optimization, enabling application-specific kernel policies previously requiring deep expertise. By bridging the gap between application needs and kernel capabilities, we enable optimal performance across diverse environments from cloud datacenters to edge devices. The self-evolving scheduler library transforms static operating systems into intelligent, adaptive infrastructure that learns and improves over time. This shifts the paradigm from manual tuning to autonomous optimization, freeing developers to focus on application logic while AI handles system-level performance.

Building trust requires comprehensive testing, audit trails, and addressing ethical considerations around fairness and privacy. We encourage community contributions to our open-source implementation, workload trace sharing for AI training, and exploration of applications beyond CPU scheduling as we transform static operating systems into intelligent, self-evolving infrastructure.
\section{Conclusion and Future Work}

\subsection{Summary}

We present the first framework for using fully automatic LLM agents to dynamically optimize Linux schedulers. By maintaining a self-evolving library of schedulers and leveraging reinforcement learning for continuous improvement, our system achieves significant performance gains while reducing the expertise required for OS optimization. 

Our key contributions include:
\begin{itemize}
\item A comprehensive framework that enables any AI agent to optimize OS schedulers
\item Design principles for AI-system interfaces that balance capability, cost, and safety
\item Demonstrated performance improvements of 30-80\% across diverse workloads
\item Cost reductions of 85-88\% compared to naive AI approaches
\item Open-source implementation proving the feasibility of AI-driven OS optimization
\end{itemize}

This work demonstrates that with proper design, AI can democratize system optimization—making expert-level performance accessible to users from cloud operators to gamers, while paving the way for truly self-optimizing operating systems.

\subsection{Impact and Vision}

Our work has implications beyond schedulers. As AI agents become more powerful, the interfaces we design today will shape how AI interacts with systems software tomorrow. We envision a future where:

\begin{itemize}
\item Every application runs on an OS perfectly tuned for its needs
\item System optimization is no longer the domain of a few experts
\item Operating systems continuously evolve and improve without human intervention
\item Performance gains compound as the scheduler library grows
\end{itemize}

The principles we establish—decoupling, context balance, composable tools, and safety-first design—are generalizable to other system components and even other domains where AI agents interact with complex systems.

\subsection{Future Work}

\subsubsection{Extended Scope}
While we focus on CPU schedulers, the framework naturally extends to other OS policies:
\begin{itemize}
\item \textbf{Memory Management}: Page replacement algorithms, NUMA policies
\item \textbf{I/O Scheduling}: Disk schedulers, network queue management
\item \textbf{Power Management}: DVFS policies, core parking strategies
\item \textbf{Security Policies}: Dynamic security configurations based on workload
\end{itemize}

\subsubsection{Cross-Component Optimization}
Future work should explore coordinated optimization across multiple system components:
\begin{itemize}
\item Joint CPU-memory scheduling for memory-intensive workloads
\item Coordinated power and performance optimization
\item System-wide policy synthesis considering all resources
\end{itemize}

\subsubsection{Industry Adoption Path}
To facilitate adoption, we plan to:
\begin{itemize}
\item Partner with cloud providers for large-scale deployment
\item Develop certification processes for AI-generated schedulers
\item Create industry standards for AI-system interfaces
\item Build trust through gradual deployment and extensive testing
\end{itemize}

\subsubsection{Enhanced AI Capabilities}
As AI models improve, our framework can leverage:
\begin{itemize}
\item Multi-modal understanding (code + performance graphs + logs)
\item Longer context windows for more comprehensive analysis
\item Faster inference for real-time optimization
\item Better code generation reducing the need for safety checks
\end{itemize}

\subsubsection{Theoretical Foundations}
Important theoretical questions remain:
\begin{itemize}
\item Formal verification of AI-generated schedulers
\item Bounds on performance improvements achievable through AI
\item Convergence guarantees for the self-evolution process
\item Optimal design of AI-system interfaces
\end{itemize}

\subsection{Challenges and Considerations}

\subsubsection{Trust and Reliability}
Building trust in AI-generated kernel code requires:
\begin{itemize}
\item Extensive testing and gradual rollout procedures
\item Clear audit trails and explainable decisions
\item Fallback mechanisms and safety guarantees
\item Industry collaboration on standards and best practices
\end{itemize}

\subsubsection{Ethical Considerations}
AI-driven optimization raises important questions:
\begin{itemize}
\item Fairness in resource allocation across different users
\item Privacy implications of workload analysis
\item Environmental impact of optimization decisions
\item Transparency in AI-driven system behavior
\end{itemize}

\subsubsection{Technical Debt}
As the scheduler library grows, we must address:
\begin{itemize}
\item Maintenance of AI-generated code
\item Evolution of the framework with changing AI capabilities
\item Backward compatibility with existing systems
\item Documentation and understanding of generated policies
\end{itemize}

\subsection{Call to Action}

This work opens new possibilities for adaptive, application-aware operating systems that can automatically optimize themselves for changing workloads. We call on the community to:

\begin{itemize}
\item Contribute to the open-source implementation
\item Share workload traces and performance data
\item Develop new safety mechanisms and verification tools
\item Explore applications to other system components
\item Help establish standards for AI-system interfaces
\end{itemize}

The era of static, one-size-fits-all operating systems is ending. With AI agents as partners, we can build systems that understand, adapt, and optimize continuously. This is not just an incremental improvement—it's a fundamental shift in how we think about system software. The future of operating systems is not just self-driving, but self-evolving, and this work takes the first crucial steps on that journey.


\bibliographystyle{plain}
\bibliography{sample-base}






\end{document}