\section{Implementation}
\label{sec:implementation}

We implement \sys as a modular system centered around an MCP server built with Python FastAPI that provides the interface between AI agents and the production system through RESTful endpoints for workload submission, scheduler retrieval, and performance monitoring, with WebSocket support for real-time metric streaming and authentication via API keys with rate limiting to prevent abuse. The scheduler library serves as institutional memory using Git for version control, JSON metadata for scheduler descriptions and performance characteristics, SQLite for indexing performance metrics across deployments, and Sentence-BERT embeddings for natural language search, with automatic indexing that extracts patterns from new schedulers and regression detection to maintain library quality. Our code generation pipeline transforms AI intent into deployable schedulers through template selection matching workload descriptions to proven patterns, parameter injection for workload-specific customization, static validation catching errors before kernel interaction, and incremental compilation with feedback loops that use errors and test results to guide AI refinement. Safety permeates every layer through defense-in-depth validation including syntax checking with clang-format, BPF verifier pre-check simulating kernel verification, resource analysis proving bounded execution, and security scanning for vulnerabilities, complemented by runtime protection with automatic fallback to CFS on anomalies, configurable performance monitoring thresholds, resource quotas preventing runaway consumption, comprehensive audit logging, and gradual rollout testing schedulers on increasing workload percentages before full deployment.