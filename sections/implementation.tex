\section{Key Technical Challenges and Solutions}

\subsection{Code Generation Efficiency}

\subsubsection{Challenge}
How can LLM agents estimate program execution time and performance characteristics from code alone? Our experiments showed that AI-generated schedulers often have unexpected performance characteristics due to subtle implementation choices.

\subsubsection{Solution}
We developed a multi-faceted approach combining static and dynamic analysis:

\textbf{Profile-Guided Optimization}:
\begin{itemize}
\item Maintain performance profiles of common code patterns
\item Use historical data to predict execution characteristics
\item Correlate code features (loop depth, data structure choice) with runtime behavior
\end{itemize}

\textbf{Lightweight Static Analysis}:
\begin{itemize}
\item Analyze eBPF instruction sequences for complexity estimation
\item Identify performance anti-patterns (excessive branching, inefficient data structures)
\item Provide early feedback before expensive runtime testing
\end{itemize}

\textbf{Incremental Refinement}:
\begin{itemize}
\item Start with conservative implementations
\item Progressively optimize based on profiling data
\item Use A/B testing to validate improvements
\end{itemize}

\subsection{Token Consumption Management}

\subsubsection{Challenge}
Large codebases and system documentation require excessive LLM tokens, making AI-driven optimization prohibitively expensive. Our initial experiments consumed over 200 API calls for a single scheduler.

\subsubsection{Solution}
We implemented intelligent context management strategies:

\textbf{Hierarchical Code Understanding}:
\begin{itemize}
\item Abstract code into high-level summaries for initial decisions
\item Drill down to implementation details only when necessary
\item Cache intermediate representations for reuse
\end{itemize}

\textbf{Focus on Critical Paths}:
\begin{itemize}
\item Identify performance-critical sections through profiling
\item Limit detailed analysis to hot paths
\item Use heuristics for non-critical code sections
\end{itemize}

\textbf{Semantic Compression}:
\begin{itemize}
\item Extract key patterns and idioms from successful schedulers
\item Represent complex implementations as parameterized templates
\item Use domain-specific abstractions to reduce token count
\end{itemize}

\textbf{Example Token Optimization}:
\begin{verbatim}
# Instead of full code:
// 500+ lines of scheduler implementation

# Use semantic summary:
{
  "type": "work_stealing_scheduler",
  "features": ["numa_aware", "priority_queues"],
  "critical_params": ["steal_threshold", "migration_cost"],
  "performance_profile": "low_latency_high_throughput"
}
\end{verbatim}

\subsection{Safety Guarantees}

\subsubsection{Challenge}
Prevent system crashes, performance degradation, and security vulnerabilities from AI-generated code. Kernel-level code errors can have catastrophic consequences.

\subsubsection{Solution}
We implemented a comprehensive safety framework:

\textbf{Static Analysis Pipeline}:
\begin{enumerate}
\item \textbf{Syntax Validation}: Ensure valid C and BPF code
\item \textbf{BPF Verifier Pre-check}: Simulate kernel verification
\item \textbf{Resource Bound Analysis}: Check for infinite loops, unbounded memory
\item \textbf{Security Audit}: Scan for common vulnerabilities
\end{enumerate}

\textbf{Sandboxed Testing Environment}:
\begin{itemize}
\item Isolated VMs for initial scheduler testing
\item Simulated workloads before real deployment
\item Performance regression detection
\item Automatic rollback on failures
\end{itemize}

\textbf{Gradual Rollout Mechanism}:
\begin{itemize}
\item Start with 1\% of workload on new scheduler
\item Monitor key metrics (latency, throughput, CPU usage)
\item Progressively increase coverage if metrics improve
\item Instant rollback on anomaly detection
\end{itemize}

\textbf{Safety Verification Example}:
\begin{verbatim}
def verify_scheduler_safety(scheduler_code):
    # Static checks
    if not passes_bpf_verifier(scheduler_code):
        return False, "BPF verification failed"
    
    # Resource bounds
    if has_unbounded_loops(scheduler_code):
        return False, "Unbounded loop detected"
    
    # Performance estimation
    overhead = estimate_overhead(scheduler_code)
    if overhead > MAX_ACCEPTABLE_OVERHEAD:
        return False, f"Overhead {overhead} exceeds limit"
    
    return True, "All safety checks passed"
\end{verbatim}

\subsection{Performance Validation}

\subsubsection{Challenge}
Ensure AI-generated schedulers match or exceed human expert performance while maintaining system stability.

\subsubsection{Solution}

\textbf{Multi-Stage Validation}:
\begin{enumerate}
\item \textbf{Microbenchmarks}: Test specific scheduler operations
\item \textbf{Synthetic Workloads}: Controlled stress testing
\item \textbf{Real Application Testing}: Actual workload performance
\item \textbf{Production Shadowing}: Run alongside existing scheduler
\end{enumerate}

\textbf{Performance Metrics Framework}:
\begin{itemize}
\item Latency percentiles (p50, p95, p99)
\item Throughput under various loads
\item CPU utilization efficiency
\item Context switch overhead
\item Cache performance impact
\end{itemize}

\textbf{Continuous Monitoring}:
\begin{itemize}
\item Real-time performance dashboards
\item Anomaly detection algorithms
\item Automatic performance reports
\item Feedback loop to AI agent
\end{itemize}

\subsection{Integration with Production Systems}

\subsubsection{Challenge}
Deploy AI-generated schedulers in production environments without disrupting existing workloads or requiring system downtime.

\subsubsection{Solution}

\textbf{Hot-Swappable Scheduler Architecture}:
\begin{itemize}
\item Leverage sched\_ext's dynamic loading capabilities
\item Zero-downtime scheduler transitions
\item State migration between schedulers
\item Fallback to default CFS on errors
\end{itemize}

\textbf{Compatibility Layer}:
\begin{itemize}
\item Support existing cgroup hierarchies
\item Preserve nice values and priorities
\item Maintain CPU affinity settings
\item Honor real-time constraints
\end{itemize}

\textbf{Operational Integration}:
\begin{itemize}
\item Logging compatible with existing monitoring
\item Metrics exported to standard observability stacks
\item Configuration through familiar interfaces
\item Documentation generated automatically
\end{itemize}

\subsection{Implementation Details}

\subsubsection{Technology Stack}
\begin{itemize}
\item \textbf{Core Framework}: Python 3.11+ with async support
\item \textbf{LLM Integration}: OpenAI, Anthropic, and local model APIs
\item \textbf{BPF Toolchain}: libbpf, bpftool, clang-15+
\item \textbf{Monitoring}: Prometheus metrics, OpenTelemetry traces
\item \textbf{Testing}: pytest, sysbench, custom workload generators
\end{itemize}

\subsubsection{Performance Optimizations}
\begin{itemize}
\item \textbf{Caching Layer}: Redis for API responses and analysis results
\item \textbf{Parallel Processing}: Concurrent scheduler testing
\item \textbf{Incremental Compilation}: Only rebuild changed components
\item \textbf{Profile-Guided Optimization}: Use runtime data for code generation
\end{itemize}

\subsubsection{Deployment Architecture}
\begin{itemize}
\item \textbf{Control Plane}: REST API for agent interaction
\item \textbf{Data Plane}: High-performance metric collection
\item \textbf{Storage}: Time-series database for historical data
\item \textbf{Orchestration}: Kubernetes operators for lifecycle management
\end{itemize}