\section{Future Work and Impact}

Our framework extends to memory management with custom page replacement policies, NUMA optimization, I/O scheduling for NVMe SSDs, and power management through intelligent DVFS policies. Cross-component optimization coordinating CPU, memory, I/O, and power decisions holistically presents significant opportunities. Research challenges include formal verification of AI-generated schedulers, understanding performance improvement bounds, and leveraging AI advances like multi-modal understanding and million-token context windows. This work democratizes OS optimization, enabling application-specific kernel policies previously requiring deep expertise. By bridging the gap between application needs and kernel capabilities, we enable optimal performance across diverse environments from cloud datacenters to edge devices. The self-evolving scheduler library transforms static operating systems into intelligent, adaptive infrastructure that learns and improves over time. This shifts the paradigm from manual tuning to autonomous optimization, freeing developers to focus on application logic while AI handles system-level performance. Building trust requires comprehensive testing, audit trails, and addressing ethical considerations around fairness and privacy. We encourage community contributions to our open-source implementation, workload trace sharing for AI training, and exploration of applications beyond CPU scheduling as we transform static operating systems into intelligent, self-evolving infrastructure.