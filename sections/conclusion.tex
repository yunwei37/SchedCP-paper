\section{Conclusion}

We present the first framework enabling LLM agents to automatically optimize Linux schedulers, achieving 30-80\% performance improvements while reducing costs by 85-88\%. Our modular design democratizes system optimization—anyone can describe workload requirements in natural language to obtain expert-level performance previously requiring kernel engineering teams. The self-evolving scheduler library continuously improves through accumulated knowledge, shifting from static software to learning systems. Built on principles of decoupling AI and system responsibilities, maintaining context balance, providing composable tools, and enforcing safety, our framework extends beyond CPU scheduling to memory management, I/O scheduling, and power optimization. This work proves AI-driven OS optimization is practical and economically viable, marking the transition from static operating systems to intelligent, self-evolving systems that optimize for their workloads.

This work has implications beyond schedulers. As AI agents become more powerful, the interfaces we design today will shape how AI interacts with systems software tomorrow. Our framework demonstrates that with proper design, AI can democratize system optimization—making expert-level performance accessible to users from cloud operators to gamers, while paving the way for truly self-optimizing operating systems.