\section{Conclusion and Future Work}

\subsection{Summary}

We present the first framework for using fully automatic LLM agents to dynamically optimize Linux schedulers. By maintaining a self-evolving library of schedulers and leveraging reinforcement learning for continuous improvement, our system achieves significant performance gains while reducing the expertise required for OS optimization. 

Our key contributions include:
\begin{itemize}
\item A comprehensive framework that enables any AI agent to optimize OS schedulers
\item Design principles for AI-system interfaces that balance capability, cost, and safety
\item Demonstrated performance improvements of 30-80\% across diverse workloads
\item Cost reductions of 85-88\% compared to naive AI approaches
\item Open-source implementation proving the feasibility of AI-driven OS optimization
\end{itemize}

This work demonstrates that with proper design, AI can democratize system optimization—making expert-level performance accessible to users from cloud operators to gamers, while paving the way for truly self-optimizing operating systems.

\subsection{Impact and Vision}

Our work has implications beyond schedulers. As AI agents become more powerful, the interfaces we design today will shape how AI interacts with systems software tomorrow. We envision a future where:

\begin{itemize}
\item Every application runs on an OS perfectly tuned for its needs
\item System optimization is no longer the domain of a few experts
\item Operating systems continuously evolve and improve without human intervention
\item Performance gains compound as the scheduler library grows
\end{itemize}

The principles we establish—decoupling, context balance, composable tools, and safety-first design—are generalizable to other system components and even other domains where AI agents interact with complex systems.

\subsection{Future Work}

\subsubsection{Extended Scope}
While we focus on CPU schedulers, the framework naturally extends to other OS policies:
\begin{itemize}
\item \textbf{Memory Management}: Page replacement algorithms, NUMA policies
\item \textbf{I/O Scheduling}: Disk schedulers, network queue management
\item \textbf{Power Management}: DVFS policies, core parking strategies
\item \textbf{Security Policies}: Dynamic security configurations based on workload
\end{itemize}

\subsubsection{Cross-Component Optimization}
Future work should explore coordinated optimization across multiple system components:
\begin{itemize}
\item Joint CPU-memory scheduling for memory-intensive workloads
\item Coordinated power and performance optimization
\item System-wide policy synthesis considering all resources
\end{itemize}

\subsubsection{Industry Adoption Path}
To facilitate adoption, we plan to:
\begin{itemize}
\item Partner with cloud providers for large-scale deployment
\item Develop certification processes for AI-generated schedulers
\item Create industry standards for AI-system interfaces
\item Build trust through gradual deployment and extensive testing
\end{itemize}

\subsubsection{Enhanced AI Capabilities}
As AI models improve, our framework can leverage:
\begin{itemize}
\item Multi-modal understanding (code + performance graphs + logs)
\item Longer context windows for more comprehensive analysis
\item Faster inference for real-time optimization
\item Better code generation reducing the need for safety checks
\end{itemize}

\subsubsection{Theoretical Foundations}
Important theoretical questions remain:
\begin{itemize}
\item Formal verification of AI-generated schedulers
\item Bounds on performance improvements achievable through AI
\item Convergence guarantees for the self-evolution process
\item Optimal design of AI-system interfaces
\end{itemize}

\subsection{Challenges and Considerations}

\subsubsection{Trust and Reliability}
Building trust in AI-generated kernel code requires:
\begin{itemize}
\item Extensive testing and gradual rollout procedures
\item Clear audit trails and explainable decisions
\item Fallback mechanisms and safety guarantees
\item Industry collaboration on standards and best practices
\end{itemize}

\subsubsection{Ethical Considerations}
AI-driven optimization raises important questions:
\begin{itemize}
\item Fairness in resource allocation across different users
\item Privacy implications of workload analysis
\item Environmental impact of optimization decisions
\item Transparency in AI-driven system behavior
\end{itemize}

\subsubsection{Technical Debt}
As the scheduler library grows, we must address:
\begin{itemize}
\item Maintenance of AI-generated code
\item Evolution of the framework with changing AI capabilities
\item Backward compatibility with existing systems
\item Documentation and understanding of generated policies
\end{itemize}

\subsection{Call to Action}

This work opens new possibilities for adaptive, application-aware operating systems that can automatically optimize themselves for changing workloads. We call on the community to:

\begin{itemize}
\item Contribute to the open-source implementation
\item Share workload traces and performance data
\item Develop new safety mechanisms and verification tools
\item Explore applications to other system components
\item Help establish standards for AI-system interfaces
\end{itemize}

The era of static, one-size-fits-all operating systems is ending. With AI agents as partners, we can build systems that understand, adapt, and optimize continuously. This is not just an incremental improvement—it's a fundamental shift in how we think about system software. The future of operating systems is not just self-driving, but self-evolving, and this work takes the first crucial steps on that journey.