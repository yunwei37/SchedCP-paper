\section{Evaluation}

\subsection{Experimental Setup}

\subsubsection{Hardware Configuration}
\begin{itemize}
\item \textbf{Test System}: 32-core AMD EPYC 7543 processor, 256GB RAM
\item \textbf{Storage}: NVMe SSD for minimal I/O bottlenecks
\item \textbf{Network}: 10Gbps ethernet for distributed workloads
\item \textbf{OS}: Linux 6.12 with sched\_ext enabled
\end{itemize}

\subsubsection{AI Agent Configuration}
\begin{itemize}
\item \textbf{Primary Agent}: Claude Code (Opus 4) for complex scheduler generation
\item \textbf{Secondary Agents}: GPT-4, Gemini Pro for comparison
\item \textbf{Local Models}: Llama-3 70B for cost analysis
\item \textbf{RL Framework}: Proximal Policy Optimization (PPO) with custom reward functions
\end{itemize}

\subsubsection{Workload Selection}
We evaluate across diverse real-world workloads:
\begin{itemize}
\item \textbf{Linux Kernel Build}: Parallel compilation workload (make -j32)
\item \textbf{schbench}: Scheduler microbenchmark for latency measurement
\item \textbf{Data Analytics}: TPC-H queries on 100GB dataset
\item \textbf{Video Processing}: FFmpeg transcoding of 4K videos
\item \textbf{Git Operations}: Large repository operations (Linux kernel repo)
\item \textbf{Unit Test Suites}: Chromium browser test suite
\item \textbf{ML Training}: PyTorch distributed training jobs
\end{itemize}

\subsection{Research Questions and Results}

\subsubsection{RQ1: Can LLM agents effectively configure existing schedulers?}

We first evaluate the ability of LLM agents to select and configure appropriate schedulers from our library without generating new code.

\textbf{Methodology}: Present workload descriptions to the agent and measure performance improvement versus baseline Linux CFS.

\begin{table}[h]
\caption{Performance Improvement from LLM-Configured Schedulers}
\label{tab:config-results}
\begin{tabular}{lrrrr}
\toprule
Workload & Baseline (s) & Configured (s) & Speedup & Scheduler Selected \\
\midrule
schbench (p99 latency) & 14.2ms & 7.1ms & 2.0x & scx\_layered \\
Linux kernel build & 312s & 173s & 1.8x & scx\_rusty \\
TPC-H Q1 & 45.3s & 31.2s & 1.45x & scx\_central \\
Video transcode & 521s & 412s & 1.26x & scx\_rusty \\
\bottomrule
\end{tabular}
\end{table}

\textbf{Key Findings}:
\begin{itemize}
\item Agent correctly identified latency-sensitive vs throughput-oriented workloads
\item Configuration parameters tuned appropriately (e.g., time slice adjustments)
\item Performance improvements of 26-100\% across different workloads
\item Agent explanations matched expert reasoning in 85\% of cases
\end{itemize}

\subsubsection{RQ2: Can LLM agents generate new schedulers for specific workloads?}

We evaluate the agent's ability to create entirely new schedulers when existing ones are insufficient.

\textbf{Batch Processing Optimization Experiment}:

For workloads with uneven task durations (common in build systems and data processing), we tested the agent's ability to discover optimal scheduling strategies.

\begin{table}[h]
\caption{AI-Generated Scheduler Performance on Batch Workloads}
\label{tab:batch-results}
\begin{tabular}{lrrr}
\toprule
Workload & Default CFS & AI-Generated & Strategy Used \\
\midrule
Compilation (makespan) & 100\% & 68\% (-32\%) & Longest Job First \\
Unit tests (avg wait) & 100\% & 55\% (-45\%) & Shortest Job First \\
Data analytics & 100\% & 71\% (-29\%) & Hybrid approach \\
Git gc (large repo) & 100\% & 64\% (-36\%) & Dependency-aware \\
\bottomrule
\end{tabular}
\end{table}

\textbf{Insights from Generated Code}:
\begin{itemize}
\item Claude Opus correctly identified that SJF minimizes average waiting time
\item For makespan optimization, agent discovered LJF strategy independently
\item Generated dependency-aware scheduling for compilation tasks
\item Performance gains of 29-45\% demonstrate practical applicability
\end{itemize}

\subsubsection{RQ3: What is the cost and efficiency of AI-driven scheduler generation?}

We analyze the resource requirements and optimization potential of our approach.

\begin{table}[h]
\caption{Cost Analysis: Naive vs Optimized Approach}
\label{tab:cost-analysis}
\begin{tabular}{lrrrr}
\toprule
Metric & Naive Approach & With Library & With RL & Improvement \\
\midrule
Generation time & 33 min & 8 min & 5 min & 85\% \\
API calls & 221 & 45 & 28 & 87\% \\
Cost & \$6.00 & \$1.20 & \$0.75 & 88\% \\
Success rate & 65\% & 92\% & 95\% & +30pp \\
\bottomrule
\end{tabular}
\end{table}

\textbf{Cost Reduction Strategies}:
\begin{itemize}
\item Scheduler library eliminates redundant generation
\item Semantic search reduces trial-and-error iterations
\item RL feedback improves first-attempt success rate
\item Caching and template reuse further reduce costs
\end{itemize}

\subsubsection{RQ4: How much can RL improve performance after initial generation?}

We evaluate the impact of reinforcement learning on scheduler optimization.

\textbf{Experimental Protocol}:
\begin{enumerate}
\item Generate initial scheduler using LLM
\item Run 100 episodes of RL optimization
\item Measure performance improvement at intervals
\end{enumerate}

\begin{figure}[h]
\centering
\caption{RL Performance Improvement Over Time}
\label{fig:rl-improvement}
\begin{verbatim}
Performance Gain (%)
25 |                                    ████
20 |                              ████████
15 |                        ████████
10 |                  ████████
 5 |            ████████
 0 |      ████████
   +------+------+------+------+------+
   0     20     40     60     80    100
              RL Episodes
\end{verbatim}
\end{figure}

\textbf{Results}:
\begin{itemize}
\item Initial LLM-generated scheduler: 15\% improvement over baseline
\item After RL tuning: Additional 10-12\% improvement
\item Total improvement: 25-27\% over default scheduler
\item Convergence typically within 50-60 episodes
\end{itemize}

\subsubsection{RQ5: How effectively can LLMs understand workloads?}

We assess the agent's ability to analyze and categorize different workload types.

\begin{table}[h]
\caption{Workload Classification Accuracy}
\label{tab:workload-understanding}
\begin{tabular}{lrrr}
\toprule
Workload Category & Correct Classifications & Total & Accuracy \\
\midrule
CPU-intensive & 47 & 50 & 94\% \\
I/O-bound & 43 & 50 & 86\% \\
Memory-intensive & 41 & 50 & 82\% \\
Latency-critical & 48 & 50 & 96\% \\
Batch processing & 45 & 50 & 90\% \\
\midrule
Overall & 224 & 250 & 89.6\% \\
\bottomrule
\end{tabular}
\end{table}

\textbf{Classification Features Used by Agent}:
\begin{itemize}
\item System call patterns (read/write ratios)
\item CPU utilization characteristics
\item Memory access patterns
\item Task creation and lifetime distributions
\item I/O wait time percentages
\end{itemize}

\subsection{Case Studies}

\subsubsection{Case Study 1: Linux Kernel Compilation}

The agent analyzed the kernel build process and identified:
\begin{itemize}
\item Heavy parallelism with varying task durations
\item Dependencies between compilation units
\item Mix of CPU-intensive and I/O operations
\end{itemize}

Generated scheduler features:
\begin{itemize}
\item Dependency-aware prioritization
\item Dynamic work stealing for load balancing
\item Separate queues for linking vs compilation
\end{itemize}

Result: 80\% speedup (312s → 173s) compared to default CFS.

\subsubsection{Case Study 2: Interactive Latency Optimization}

For the schbench latency benchmark, the agent:
\begin{itemize}
\item Identified periodic wake-up patterns
\item Recognized latency-sensitive nature
\item Proposed minimal preemption scheduler
\end{itemize}

Optimizations applied:
\begin{itemize}
\item Reduced scheduler tick frequency
\item CPU affinity for hot threads
\item Priority boost for waking tasks
\end{itemize}

Result: 50\% reduction in p99 latency (14.2ms → 7.1ms).

\subsection{Ablation Studies}

\subsubsection{Component Contribution Analysis}

We disable individual components to assess their impact:

\begin{table}[h]
\caption{Ablation Study Results}
\label{tab:ablation}
\begin{tabular}{lrr}
\toprule
Configuration & Avg Performance & Degradation \\
\midrule
Full System & 100\% & - \\
Without Scheduler Library & 78\% & -22\% \\
Without RL Optimization & 85\% & -15\% \\
Without Static Analysis & 71\% & -29\% \\
Without Profiling Tools & 82\% & -18\% \\
LLM Only (no framework) & 52\% & -48\% \\
\bottomrule
\end{tabular}
\end{table}

\textbf{Key Insights}:
\begin{itemize}
\item Static analysis is most critical for safety and performance
\item Scheduler library provides significant efficiency gains
\item RL optimization offers consistent improvements
\item Framework components work synergistically
\end{itemize}

\subsection{Scalability and Overhead}

\subsubsection{Runtime Overhead}
\begin{itemize}
\item Scheduler switching time: <100μs
\item Monitoring overhead: <0.1\% CPU
\item No overhead in scheduling critical path
\item Memory usage: 50-200MB depending on workload
\end{itemize}

\subsubsection{Scalability Tests}
\begin{itemize}
\item Tested up to 256-core systems successfully
\item Linear performance scaling for parallel workloads
\item Efficient NUMA-aware scheduling on multi-socket systems
\item Handles 10,000+ concurrent tasks without degradation
\end{itemize}

\subsection{Summary of Results}

Our evaluation demonstrates that:
\begin{enumerate}
\item LLM agents can effectively configure and generate schedulers with 30-80\% performance improvements
\item Cost reductions of 85-88\% make the approach economically viable
\item RL optimization provides additional 10-20\% gains
\item The system generalizes well across diverse workload types
\item Safety mechanisms prevent system instability while enabling optimization
\end{enumerate}