\section{Evaluation}

\subsection{Research Questions}

We investigate five key research questions to validate the effectiveness and efficiency of \sys:

\begin{itemize}
\item \textbf{RQ1}: Can \sys effectively configure existing schedulers?
\item \textbf{RQ2}: Can \sys generate new schedulers for specific workloads?
\item \textbf{RQ3}: What is the cost and efficiency of \sys's scheduler generation?
\item \textbf{RQ4}: How much can RL improve performance after initial generation?
\item \textbf{RQ5}: How effectively can \sys understand workloads?
\end{itemize}

\subsection{Experimental Setup}

We evaluate \sys through comprehensive experiments designed to validate LLM capabilities in scheduler optimization. Our testbed consists of a 32-core AMD EPYC 7543 processor with 256GB DDR4-3200 RAM, representing modern server deployments. The EPYC platform's chiplet architecture with multiple CCX units connected via Infinity Fabric presents realistic NUMA challenges. Storage uses enterprise NVMe SSDs eliminating I/O bottlenecks, focusing tests on scheduler behavior. All systems run Linux 6.12 with sched\_ext enabled. We test multiple AI agents including Claude Code (Opus 4), GPT-4, and Gemini Pro to evaluate framework generality, with Llama-3 70B for cost-sensitive deployments. Our workload suite spans latency-sensitive applications (schbench web service simulation, database queries), throughput-oriented tasks (Linux kernel compilation, video transcoding), batch processing scenarios (data analytics, log processing, unit testing), and mixed workloads (Git operations, Chromium test suite).

\subsection{Performance Impact of AI-Driven Optimization}

Figure~\ref{fig:performance-comparison} demonstrates the multi-stage performance improvements achieved through \sys, addressing both RQ1 (LLM configuration effectiveness) and RQ4 (RL enhancement impact). We evaluate performance across diverse workloads comparing three configurations: baseline Linux CFS scheduler, LLM-configured schedulers where the agent selects and tunes parameters, and RL-improved schedulers after feedback loop optimization.

\begin{figure}[h]
\centering
\fbox{\parbox{0.9\columnwidth}{\centering
\vspace{2cm}
Performance Comparison: Baseline vs LLM vs RL-Enhanced\\
(Grouped bar chart showing 30-80\% LLM gains + 10-12\% RL improvement)\\
\vspace{2cm}
}}
\caption{Multi-stage performance improvements from baseline to RL-enhanced schedulers. Shows initial LLM configuration gains (30-80\%) and additional RL improvements (10-12\%) across different workload types.}
\label{fig:performance-comparison}
\end{figure}

The results show significant performance improvements through intelligent scheduler selection and configuration. For schbench, the LLM achieves 50\% lower p99 latency and 30\% higher throughput by selecting scx\_layered with minimal time slices and aggressive priority boosting. Linux kernel build demonstrates approximately 80\% speedup, reducing compilation time from 312s to 173s through scx\_rusty with work-stealing enabled. The LLM correctly identifies latency-sensitive versus throughput-oriented workloads in 85\% of cases, matching expert recommendations. Reinforcement learning adds substantial value beyond initial LLM configuration, contributing an additional 10-12\% performance gain through parameter fine-tuning and workload-specific pattern discovery. Total improvements reach 25-27\% over baseline CFS, with RL convergence typically occurring within 50-60 episodes. The RL component discovers optimization opportunities missed by initial generation, such as optimal migration thresholds and time slice adjustments based on actual workload behavior.

\begin{table}[h]
\caption{Performance Improvements Across Workload Types}
\label{tab:performance-results}
\begin{tabular}{lrrrr}
\toprule
Workload & Baseline & LLM Config & RL-Enhanced & Total Gain \\
\midrule
schbench (p99 latency) & 14.2ms & 7.1ms & 6.3ms & 56\% \\
Linux kernel build & 312s & 173s & 156s & 50\% \\
TPC-H Q1 & 45.3s & 31.2s & 28.9s & 36\% \\
Video transcode & 521s & 412s & 385s & 26\% \\
\bottomrule
\end{tabular}
\end{table}

\subsection{Novel Scheduler Synthesis for Batch Workloads}

Beyond configuration, \sys demonstrates the ability to generate entirely new schedulers tailored to specific workload characteristics, addressing RQ2. We focus on batch processing scenarios with uneven process workloads that challenge traditional fair-share schedulers: data analytics with join table operations, log analysis pipelines, video editing workflows, Git operations on large repositories, and unit testing suites with varying test durations.

\begin{figure}[h]
\centering
\fbox{\parbox{0.9\columnwidth}{\centering
\vspace{1.5cm}
AI-Generated Scheduler Performance\\
(Grouped bars: CFS vs AI-Generated with strategy labels)\\
\vspace{1.5cm}
}}
\caption{Performance of AI-generated schedulers on batch workloads. Shows 30-50\% improvements through theoretically optimal algorithm selection (SJF, LJF, Hybrid).}
\label{fig:batch-performance}
\end{figure}

The AI demonstrates sophisticated understanding of scheduling theory in practice. For unit tests minimizing average wait time, Claude Opus correctly implements Shortest Job First (SJF) scheduling, achieving 45\% reduction in average waiting time. For compilation workloads where total makespan matters, the agent implements Longest Job First (LJF) scheduling with 32\% improvement. The generated code includes dependency-aware scheduling for build systems, analyzing the build graph to prioritize critical path tasks. Performance improvements of 30-50\% across batch workloads validate that theoretical insights translate to practical benefits. The AI's ability to identify and implement optimal strategies varies by model—Claude Opus consistently recognizes these patterns while other models may default to simpler heuristics.

\begin{table}[h]
\caption{AI-Generated Scheduler Performance on Batch Workloads}
\label{tab:batch-results}
\begin{tabular}{lrrr}
\toprule
Workload & Default CFS & AI-Generated & Strategy Used \\
\midrule
Compilation (makespan) & 100\% & 68\% (-32\%) & Longest Job First \\
Unit tests (avg wait) & 100\% & 55\% (-45\%) & Shortest Job First \\
Data analytics & 100\% & 71\% (-29\%) & Hybrid approach \\
Git gc (large repo) & 100\% & 64\% (-36\%) & Dependency-aware \\
\bottomrule
\end{tabular}
\end{table}

\subsection{Cost Reduction and Optimization Efficiency}

Economic viability is critical for practical adoption of AI-driven optimization (RQ3). Our motivation experiments revealed prohibitive costs for naive approaches: basic FIFO scheduler generation required 33 minutes, 221 API calls, costing approximately \$6. \sys dramatically reduces these costs through multiple optimizations.

\begin{figure}[h]
\centering
\fbox{\parbox{0.9\columnwidth}{\centering
\vspace{1.5cm}
Cost Reduction Through Framework Optimizations\\
(Multi-line graph: Time, API calls, Cost, Success rate)\\
\vspace{1.5cm}
}}
\caption{Cost reduction achieved through scheduler library, semantic search, and RL optimization. Shows 85-88\% reduction in time, API calls, and monetary cost.}
\label{fig:cost-reduction}
\end{figure}

The scheduler library serves as institutional memory, eliminating redundant generation—60\% of requests are satisfied through library adaptation rather than full generation. Semantic search through our pattern database provides relevant examples, cutting average iteration count from 15+ to under 3. Caching API responses for documentation queries reduces calls by 40\%. Template reuse lets AI focus on customization rather than boilerplate. These optimizations achieve 85-88\% cost reduction: generation time drops from 33 to 5 minutes, API calls decrease from 221 to 28, and cost falls from \$6.00 to \$0.75. Success rate improves from 65\% to 95\% through learned patterns and better prompting strategies, making AI-driven scheduler optimization economically competitive with human development.

\begin{table}[h]
\caption{Cost Analysis: Naive vs Optimized Approach}
\label{tab:cost-analysis}
\begin{tabular}{lrrrr}
\toprule
Metric & Naive & With Library & With RL & Improvement \\
\midrule
Generation time & 33 min & 8 min & 5 min & 85\% \\
API calls & 221 & 45 & 28 & 87\% \\
Cost & \$6.00 & \$1.20 & \$0.75 & 88\% \\
Success rate & 65\% & 92\% & 95\% & +30pp \\
\bottomrule
\end{tabular}
\end{table}

\subsection{Workload Classification and Understanding}

AI-driven scheduler optimization depends on LLMs accurately understanding workload characteristics from high-level descriptions and system observations (RQ5). We evaluate classification accuracy across diverse workload categories to validate this capability.

\begin{figure}[h]
\centering
\fbox{\parbox{0.9\columnwidth}{\centering
\vspace{2cm}
Workload Classification Confusion Matrix\\
(Heatmap: CPU, I/O, Memory, Latency-critical, Batch)\\
Overall Accuracy: 89.6\%\\
\vspace{2cm}
}}
\caption{Confusion matrix showing LLM workload classification accuracy. Achieves 89.6\% overall accuracy with best performance on latency-critical (96\%) and CPU-intensive (94\%) workloads.}
\label{fig:workload-classification}
\end{figure}

LLMs achieve 89.6\% overall classification accuracy through sophisticated feature extraction. The agent examines system call patterns, inferring high read/write ratios indicate I/O-bound workloads while syscall-light applications are CPU-intensive. It analyzes CPU utilization patterns, distinguishing sustained high usage (compute-bound) from bursty patterns (interactive). Memory access patterns identify cache-sensitive workloads benefiting from NUMA-aware scheduling. Task creation and lifetime distributions reveal workload types—short-lived tasks with high creation rates suggest compilation, while long-lived tasks indicate services. Classification cost averages \$0.15 per workload analysis, making continuous optimization economically feasible. The 89.6\% accuracy validates natural language as a viable interface for scheduler optimization, bridging the gap between high-level descriptions and low-level system behaviors.

\begin{table}[h]
\caption{Workload Classification Accuracy by Category}
\label{tab:workload-classification}
\begin{tabular}{lrrr}
\toprule
Workload Category & Correct & Total & Accuracy \\
\midrule
Latency-critical & 48 & 50 & 96\% \\
CPU-intensive & 47 & 50 & 94\% \\
Batch processing & 45 & 50 & 90\% \\
I/O-bound & 43 & 50 & 86\% \\
Memory-intensive & 41 & 50 & 82\% \\
\midrule
Overall & 224 & 250 & 89.6\% \\
\bottomrule
\end{tabular}
\end{table}

